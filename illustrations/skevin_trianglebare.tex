\begin{subfigure}[t]{0.29\textwidth}
		{\centering
		\vphantom{Loss of the lexical variant}\textbf{Older\\ speakers}\\[1em]
		\DumplirOldSpeakers{0.3}}
		
		\emph{Dumplir} `a wooden candlestick carried during a funeral' is not in use anymore, although the older speakers still have a mental picture of it in their minds and are able to describe it.
	\end{subfigure}%
	\hspace{.05\textwidth}%
	\begin{subfigure}[t]{0.29\textwidth}
		{\centering
		\vphantom{Loss of the lexical variant}\textbf{Young\\ adults}\\[1em]
		\DumplirYoungAdults{0.3}}
		
		Over time, there will not be any speakers who can describe it (unless it has been described in some written text such as a dialectal glossary).
	\end{subfigure}%
	\hspace{.05\textwidth}%
	\begin{subfigure}[t]{0.29\textwidth}
		{\centering
		\vphantom{Loss of the lexical variant}\textbf{Loss of the lexical variant}\\[1em]
		\DumplirLossOfVariant{0.3}}
		
		When an object has disappeared, the addresser and the addressee cannot communicate, the object as a semiological sign cannot have any cognitive effect on the addressee, and it is no longer possible to close the circle of semiosis by  finding exclusively the same interpretant at both ends of the communication process.
		Meaning without communication is not possible, so, over time, the word disappears as well.
	\end{subfigure}%