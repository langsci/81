\documentclass[output=paper]{LSP/langsci}  
\author{Keiko Hirano\affiliation{University of Kitakyushu, Japan}} 
\title{Code-switching in the Anglophone community in Japan}
\abstract{The present study investigates code-switching in native speakers of English (NSsE) who live in an Anglophone community in Japan and examines the impact of the speakers’ social networks on their use of code-switching in a language contact situation. Sets of natural, spontaneous conversations in English between two NSsE from the same country were collected from the same informants on two separate occasions a year apart. The linguistic variable focused upon is Japanese words and phrases that could easily and naturally be expressed in English. More than 1200 Japanese words and phrases were observed in the linguistic data from 39 NSsE living in Japan as English teachers. Statistical analyses revealed that there was a significant correlation between the speaker’s use of Japanese and his/her social networks with English teachers (both Japanese and NSsE). The analysis of social networks with linguistic behaviour suggests that their identity as being not just that of ``foreigner teaching English in Japan'', but rather ``English teacher within a team made up of both Japanese and native-speaker English teachers'' is likely to encourage high frequency in the use of code-switching among the NSsE in Japan.

}

\maketitle 
\begin{document}
   

% \author{//}
 
\section{Introduction}
The purpose of this study is to investigate \isi{code-switching} in relation to \isi{social network}s in native speakers of English (NSsE) who live in an Anglophone community in Japan. The members of this Anglophone community start forming new \isi{social network}s with NSsE from their home country and other countries as well as local Japanese people as soon as they arrive in Japan. The current study observes the use of Japanese words and phrases during conversations in English between NSsE and examines the impact of the NSsE’s \isi{social network}s formed in Japan on their use of \isi{code-switching} in a \isi{language contact} situation.

Sets of natural, spontaneous conversations in English between two NSsE from the same country were collected from the same informants on two separate occasions a year apart. Each informant was also interviewed to collect information about people with whom he/she has a close relationship and regular contact in order to define his/her \isi{social network}s. The linguistic variable focused upon is Japanese words and phrases (except proper nouns) that could easily and naturally be expressed in English. The frequency of usage of such Japanese vocabulary is examined. Statistical analyses revealed that there was a significant correlation between the speaker’s use of Japanese and his/her \isi{social network}s with English teachers\is{English teacher} (both Japanese and NSsE). Analysis of \isi{social network}s with linguistic behaviour suggests that a sense of solidarity is likely to encourage high frequency in the use of \isi{code-switching} among the NSsE in Japan.

\section{Code-switching}
NSsE who come to Japan as English teachers\is{English teacher} are in a bilingual situation with English as their mother tongue and Japanese as their second language (\textbf{L2}). In such a \isi{language contact} situation, \isi{code-switching} between the two languages is likely to occur during conversation. Code-switching means that ‘bilingual or bidialectal speakers switch back and forth between one language or dialect and another within the same conversation’ \citep[23]{trudgill_glossary_2003}. According to \citet[26]{azuma_shakai_1997}, \isi{code-switching} only occurs if the interlocutor is capable of speaking the two languages at the same level as the speaker. He says that one possible reason for \isi{code-switching} occurring is the speakers having \textit{dual identities} (29-30). In performing \isi{code-switching}, the speakers attempt to confirm the fact that they both belong to dual societies synchronically and establish membership between them. It is assumed that NSsE in Japan have dual identities: their original membership and a new membership as an English teacher in Japan.\is{English teacher} For the present study, the occasional use of Japanese words and phrases during conversation in English between NSsE, as shown in Examples (1) and (2), is considered to be \isi{code-switching}.

\ea
\gll {Cathay ... it’s ... like twelve} {\textit{man}} {or something}\\
{~} {(ten thousand)} {~}\\
\z

%\begin{exe}
%\label{ex:1}
%%\langinfo{}{}
%\ex Cathay ... it’s ... like twelve \textit{man} (ten thousand) or something
%\end{exe}

\ea
\gll {there’re about \ldots ten when we had the} {\textit{eikai-}}  {not the} {\textit{eikaiwa}}  {the} {\textit{enkai}} \\
{~} {(English conver-\ldots)} {~} {(English conversation)} {~} {(party)}\\
\z

%\begin{exe}
%\label{ex:2} 
%\langinfo{}{}
%\ex there’re about ... ten when we had the \textit{eikai-} (English conver- ...) not the \textit{eikaiwa} (English conversation) the \textit{enkai} (party)
%\end{exe}

\section{Hypotheses}
In order to verify the \isi{code-switching} behaviour of NSsE living in Japan, this paper proposes two hypotheses: (1) Due to long-term \isi{language contact} with the Japanese language, \isi{code-switching} to Japanese occurs among NSsE in Japan during conversations in English between NSsE more frequently one year after their arrival in Japan; and (2) The speaker’s use of \isi{code-switching} is strongly correlated to his/her \isi{social network}s with native-speaker (\textbf{NS}) English teachers.\is{English teacher} The second hypothesis assumes that the speakers who have strong social networks with NS English teachers like themselves tend to use Japanese words and phrases more frequently than those whose comparable networks are weaker.

\section{Methodology}
\subsection{Informants and data collection}
The data used for the present study were collected from thirty-six language teachers on the Japan Exchange and Teaching ({JET}) Programme, which is sponsored by Japanese ministries \citep{council_of_local_authorities_for_international_relations_[clair]_jet_2013}, and three conversation instructors at private institutions. Fifteen English informants (5 males and 10 females), 11 Americans (7 males and 4 females), and 13 New Zealanders (3 males and 10 females) -- a total of 39 NSsE -- participated in the data collection. The informants were aged between 21 and 34 at the time of the first data collection, averaging 25 years of age. They all lived in Kyushu, mainly in the prefecture of Fukuoka and the surrounding prefectures of Kumamoto and Saga.

In order to examine linguistic change\is{linguistic change}s observed over a period of one year from arrival in Japan, the data used for this study were collected from the same informants on two separate occasions -- immediately after the informants’ arrival in Japan (2000) and then one year later (2001). The current research used a method designed to elicit more naturally occurring conversation from the informants. The interviewer was not present while the informants were being recorded in order to lessen the possibility of speech modification that might result from the presence of a researcher from Japan who is a non-NSE. In both sessions, casual conversations between two NSsE from the same country were recorded for 45 minutes. The data used for the present study comprised a total of 34 hours of speech.

For the purposes of this study, Japanese words and phrases used by the informants during the conversation in English were extracted and analysed. Proper nouns such as those shown in Examples (3) and (4) were excluded from the data. For the analysis, only Japanese words and phrases that could be expressed in English as shown in Examples (5) to (7) were included. The study includes 487 Japanese words and phrases that could be expressed in English in the first dataset and 759 Japanese words and phrases in the second dataset.

%\begin{exe}
%\label{ex:3}
%\ex I stayed at the youth hostel ... by \textit{Kawaguchiko}\textit{ (Lake Kawaguchi)} it's a lake at the base of \textit{Fuji}.
%\end{exe}

\ea
\gll {I stayed at the youth hostel \ldots by} {\textit{Kawaguchiko}} it's a lake at the base of \textit{Fuji}.\\
{~} {(Lake Kawaguchi)} {~}\\
\z

\begin{exe}
\label{ex:4}
%\langinfo{}{}
\ex And then we're going to \textit{Arita} again on Mon- on the twentieth to look at the ... you know the \textit{Kakiemon} factory
\end{exe}

%\begin{exe}
%\label{ex:5}
%\langinfo{}
%\ex I’m sure you were there when we were in the big room in the \textit{kencho} \textit{(prefectural office)}
%\end{exe}

\ea
\gll {I’m sure you were there when we were in the big room} {in the} {\textit{kencho}}\\
{~} {~} {(prefectural office)}\\
\z

%\begin{exe}
%\label{ex:6}
%\langinfo{}{}
%\ex there's a \textit{Monbusho}\textit{ } scholarship isn't there
%\end{exe}

\ea
\gll {there's a} {\textit{Monbusho}} {scholarship isn't there}\\
{~} {(Ministry of Education)} {~}\\
\z

%\begin{exe}
%\label{ex:7}
%\langinfo{}{}
%\ex my \textit{kyoto-sensei} \textit{(vice principal)}'s really nice actually
%\end{exe}

\ea
\gll {my} {\textit{kyoto-sensei}} {'s really nice actually}\\
{~} {(vice principal)} {~}\\
\z

\subsection{Social network}
The current study investigates an Anglophone community in Japan which consists of NSsE who are living temporarily in Japan as English teachers\is{English teacher} on the JET Programme and at private institutions, and who mix with speakers of different regional varieties of English in an L2 setting. Currently over 4,000 university graduates from about 40 countries participate in the JET Programme \citep{council_of_local_authorities_for_international_relations_[clair]_jet_2013}. They form relationships with people from a wide range of social contexts such as English speakers of different dialects and non-NSsE including Japanese. Thus they create a community in a new linguistic environment which differs vastly from those in their home countries.

The influence of speakers’ strong \isi{social network}s on their linguistic behaviour has been studied by researchers including \citet{cheshire_variation_1982}, \citet{eckert_adolescent_1988}, \citet{hirano_dialect_2013}, \citet{labov_language_1972} and \citet{milroy_language_1987}. Their studies revealed that there was a strong relationship between speakers’ network structures\is{network structure} and their linguistic behaviour. \citet{milroy_language_1987} studied communities in Belfast whose social networks were close-knit. Using the degree of density\is{density} of the network and the multiplexity\is{multiplexity} of each tie, she measured the strength of networks. Members of the Anglophone community in Japan, however, are socially and geographically mobile, and are always in multilingual and multidialectal contact situations. They create many network ties that form ramifying structures but their networks are loose-knit due to their relatively short stay.

In order to gather information about the social networks that the informants for the present study had created in Japan they had a short interview with the researcher and were asked about their close friends at the end of the second data collection session. The present study took into account their self-assessed closeness to other Anglophones or with Japanese, the frequency of contact with them, and network size, and developed a number of quantitative indices.\footnote{See \citet{hirano_dialect_2013} for a detailed description of the index scores of networks.} A score for each relationship was calculated using the rank order of closeness and the frequency of actual and virtual contacts with the person as follows:

Score for each relationship =   rank order score × (score for meeting frequency + score for             telephone call frequency)

These individual relationship scores were then grouped into different \isi{social network} categories. For this paper, the \isi{social network} of each informant was first grouped into two networks -- a network with NSsE and a network with non-\is{non-native speakers of English}NSsE -- and then further divided into seven sub-groups as shown in \tabref{tab:1}. Network with English teachers (7) combines network strength with NS English teachers\is{English teacher} (2) and Japanese teachers of English (6) to create another network group. These index scores of \isi{social network}s were used to examine relationships with individual informants’ frequency of \isi{code-switching} and shifts between the two datasets.

\begin{table}
\begin{tabular}{lll}
\lsptoprule
Network Members & \multicolumn{2}{l}{Social Networks}\\
\midrule
NSsE & (1) & Native speakers of English\\
& (2) & Native-speaker English teachers\\
Non-NSsE & (3) & Japanese people\\
& (4) & \begin{minipage}[t]{0.6\textwidth}Japanese who use English as their main language in speaking with the informants\end{minipage}   \\
& (5) & \begin{minipage}[t]{0.6\textwidth}Japanese who use Japanese as their main language in speaking with the informants (JJML)\end{minipage}\\
& (6) & Japanese teachers of English\\
English Teachers & (7) & English teachers (2) + (6)\\
\lspbottomrule
\end{tabular}
\label{tab:1}
\caption{Types of Social Networks}
\end{table}

\section{Results}
Multiple regression\is{multiple regression} (stepwise method) was performed to analyse the correlation between frequency in the use of Japanese words and phrases\is{flap} in individual informants a year after their arrival in Japan as the dependent variable and their \isi{social network} strengths with NSsE and non NSsE as independent variables. The result suggests that the network with English teachers\is{English teacher} and the network with Japanese who use Japanese as their main language in speaking with the informants (JJML) are statistically significantly influencing the level of Japanese usage positively, as shown in \tabref{tab:2}. The stronger such networks the informants have, the more they tend to insert Japanese words and phrases into their conversations in English with another NSE a year after their arrival in Japan. 

\tabref{tab:3} shows the result of multiple regression\is{multiple regression} analysis between the change in frequency in use of Japanese words and phrases from the first dataset to the second dataset \is{flap}in individual informants and their \isi{social network} strengths. The result suggests that the network with English teachers\is{English teacher} is the only statistically significant predictor influencing the level of Japanese usage positively. The stronger this network is, the more informants tend to increase use of Japanese in their conversations in English a year after their arrival in Japan.

According to \isi{Pearson correlation} analysis, other \isi{social network}s, such as the network with NSsE and the network with NS English teachers\is{English teacher}, showed strong correlations with the level of Japanese usage in the second dataset or with the change between the two datasets. Those networks, however, are not significant predictors according to multiple regression anayses.

\begin{table}
\begin{tabular}{lll}
\lsptoprule
Predictor Variables & Beta & p \\
\midrule
English teachers NW & .509 & .001 \\
JJML NW & .338 & .017\\
\lspbottomrule
\end{tabular}
\caption{Multiple regression for use of Japanese and social networks after a year in Japan. Adjusted R2=.344; F2,36=9.425; \textstyleSubtleEmphasis{p}=.001 (Stepwise method).}
\label{tab:2}
\end{table}

\begin{table}
\begin{tabular}{lll}
\lsptoprule
Predictor Variable & Beta & \textstyleSubtleEmphasis{p}\\
\midrule
English teachers NW & .319 & .048\\
\lspbottomrule
\end{tabular}
\caption{Multiple regression for change in frequency in use of Japanese and social networks. Adjusted R2=.078; F1,37=4.193; \textstyleSubtleEmphasis{p}=.048 (Stepwise method).}
\label{tab:3}
\end{table}

\section{Discussion}
The increase in the amount of \isi{code-switching} performed by the informants from 487 cases in the first dataset to 759 in the second dataset seems to verify the first hypothesis that "due to long-term \isi{language contact} with the Japanese language, \isi{code-switching} to Japanese occurs among NSsE in Japan during conversations in English with other NSsE more frequently one year after their arrival in Japan". The second hypothesis that "the speaker’s use of \isi{code-switching} is strongly correlated to his/her \isi{social network}s with NS English teachers" was partly verified. The results above showed that the strongest network effect on \isi{code-switching} was the one with English teachers which combines networks with NS English teachers and Japanese teachers of English. This combined network effect was much stronger than individual network effects. The informants for the current study have possibly established their identity as being not just that of 'foreigner teaching English in Japan'; but rather 'English teacher and contributor to English education in Japan within a team made up of both Japanese and NS English teachers'.\is{English teacher}

One of the possible reasons why the informants increased their frequency of usage of Japanese that could be easily expressed in English may be explained by the concept of \textstyleSubtleEmphasis{\isi{group phraseology}}. One of the applications of group phraseology offered by \citet[8]{yonekawa_shudango_2009} is its usage by certain functional social communities as a language for professional groups. The informants of the current study were all teaching English at schools. A large portion of words and phrases used by them are actually work-related and are mutually understandable even in Japanese. It might be easier to use and to understand the work-related terms in Japanese rather than those translated into English for NS English teachers when the conversation interlocutor is also in the same profession.\is{English teacher}

Another useful concept which might help to explain the linguistic behaviour of this particular social group is \textstyleSubtleEmphasis{\isi{community of practice}} \citep{eckert_adolescent_1988}. This concept is described as being active amongst ‘people who share a concern, a set of problems, or a passion about a topic, and who deepen their knowledge and expertise in this area by interacting on an ongoing basis’ \citep[4]{wenger_cultivating_2002}. NSsE who have come to Japan voluntarily to teach English are supposed to belong to a \isi{community of practice}. Solidarity or competition among NSsE could be other possible reasons for this linguistic behaviour. Further investigation will hopefully help to reveal the mechanisms of \isi{code-switching} in the Anglophone community in Japan.

\section*{Acknowledgement}
This work was supported by JSPS KAKENHI Grant Number 25284082.

\printbibliography[heading=subbibliography,notkeyword=this]
\end{document}