\documentclass[output=paper]{LSP/langsci}  
\author{Ichiro Ota\affiliation{Kagoshima University}\and Hitoshi Nikaido\affiliation{Fukuoka Jo Gakuin University} \lastand Akira Utsugi\affiliation{Nagoya University}} 
\title{Tonal Variation in Kagoshima Japanese and Factors of Language Change}
\abstract{According to \citet{kubozono_tonal_2007}, tonal changes are in progress among young native speakers of Kagoshima Japanese due to the influence of Standard Japanese through mass media broadcasting. This report presents the results of statistical analyses which demonstrate that variables related to mass media broadcasting  (media content viewing habits) seem to have positive effects on the tonal changes. In addition, we suggest a tentative theory that the asymmetrical progress of the tonal changes are reflections of de-dialectization and de-standardization taking place in this variety.
}

\maketitle 
\begin{document}


% \begin{multicols}{3}
% \centering
% Ichiro Ota\\
% \\ 
% \vskip11pt
% Hitoshi Nikaido\\
% \\
% \vskip11pt
% Akira Utsugi\\
% \\
% \end{multicols}
 

\section{Theoretical background and the research aim}
     In many dialects of Japanese, two types of word tone are recognized in terms of the accentuation. One is the tone called 'accented', which has an abrupt pitch fall within the prosodic boundary of a word. The other is 'unaccented', which has no pitch fall. In Standard Japanese (henceforth SJ), for example, the tone of \textit{na'mida-ga} 'eyedrop-NOM', with a pitch fall on the first syllable, is accented, and that of \textit{sakana-ga} 'fish-NOM', with no pitch fall, is unaccented. In addition, SJ is considered to have 'n(= number of moras) + 1' tonal patterns. Thus, for three-syllable words, there are four possible tonal patterns in the surface forms. (See the 'Tokyo' section of \tabref{tab:1}). \footnote{ The marker of prime “ ’ ” attached to words indicates the location of pitch fall.}

     Kagoshima Japanese (henceforth KJ), which is one of the provincial dialects of Japanese, has the same accentuation system SJ. However, KJ allows only two types of tone, Tone A (with a pitch fall on the penultimate syllable) and Tone B (with no pitch fall), no matter how many syllables a word has within its prosodic boundary.\footnote{ The prosodic structure of nouns consists of the word itself and a case-marking particle, such as -\textit{ga} 'NOM' and -\textit{o} 'OBJ'.} Thus, for the three-syllable words in \tabref{tab:1}, only either Tone A ( ... HL) or Tone B ( ... LH) is assigned. \footnote{ In fact, there is another interpretation for these tonal patterns \citep{kubozono_tonal_2007}. For example, Hirayama (1957) %NO REFERENCE GIVEN
considers that the difference of tones is due to 'the location of high tone' \citep[327]{kubozono_tonal_2007}. However, following Kubozono, we adopt the analysis based on the accentuation proposed by Haraguchi (1977) and Shibatani (1990) %AGAIN, NO REFERENCE
as our theoretical presupposition, because the tonal change of KJ seems to involve the presence/absence of pitch fall.  
} 
 
\begin{table} 
\resizebox{\textwidth}{!}{
\begin{tabular}{lccccc}
\lsptoprule
& \textbf{word} & namida (ga) & kokora (ga) & otoko (ga) & sakana (ga) \\
& \textbf{gloss} & eyedrop-(NOM) & heart-(NOM) & man-(NOM) & fish-(NOM)\\
\cmidrule{2-6}
\multirow{3}{.05\textwidth}{\rotatebox[origin=c]{90}{\textbf{Tokyo}}} & \textbf{surface tone} & \textit{na'mida (ga)} & \textit{koko'ro (ga)} & \textit{otoko' (ga)} & \textit{sakana ga}\\
& & HLL-(L) & LHL-(L) & LHH-(L) & LHH-(H)\\
& \textbf{accentuation} & initially accented & medially accented & finally accented & unaccented\\
\cmidrule{2-6}
\multirow{4}{.05\textwidth}{\rotatebox[origin=c]{90}{\textbf{Kagoshima}}} & \textbf{surface tone} & \textit{namida-(ga)} & \textit{kokoro-(ga)} & \textit{otoko-(ga)} & \textit{sakana'-(ga)}\\
& & LLL-(H) & LLL-(H) & LLL-(H) & LLH-(L)\\
& \textbf{accentuation} & \multicolumn{3}{c}{unaccented (...LH)} & accented (...HL)\\
& & \multicolumn{3}{c}{[Tone B]} & [Tone A] \\
\lspbottomrule
\end{tabular}
}
\caption{Tonal correspondence between TJ and KJ, adapted from \citet[329]{kubozono_tonal_2007}.}
\label{tab:1}
\end{table}

In addition, there is a sharp discrepancy between SJ and KJ in auditory impression. This is caused by a disagreement in accentuation pattern. For example, for the words in \tabref{tab:2}, the SJ tone of \textit{mo'miji} is accented (HLL), while the KJ counterpart is unaccented (LLH). This is also the case for \textit{kaede }(LHH), although their accentuations are the opposite.

However, presumably to resolve this discrepancy, tonal changes are in progress in KJ \citep{kubozono_tonal_2007}. Young native speakers of KJ tend to pronounce words with the same accentuation pattern as that of SJ, although the two-type tone system is still sufficiently preserved. Thus, a traditionally unaccented word like \textit{momiji }is likely to be produced with the accented tone (LHL), whereas \textit{kaede,} a traditionally accented word, can be realized as unaccented (LLH), as shown in the column ‘Kagoshima innovative’ of \tabref{tab:2}. Their surface forms are not perfectly identical to the TJ counterparts (HLL and LHH, respectively), but their accentuation patterns (i.e., the presence/absence of pitch fall) correspond to those of TJ. \citet[348]{kubozono_tonal_2007} claims that this tonal change is 'the result of interaction of a phonetic (or perceptual) factor' (i.e., speakers' sensitivity to the pitch accentuation) and 'a phonological factor' (i.e., the native prosodic system), and suggests a relation with the bilingualism of KJ speakers which results from the exposure 'to standard Tokyo Japanese through TV, radio and other mass media', with social dominance of SJ as the backdrop of this innovation \citep[323]{kubozono_tonal_2007}.
 
\begin{table} 
\resizebox{\textwidth}{!}{
\begin{tabular}{ccp{0.25\textwidth}p{0.25\textwidth}p{0.25\textwidth}}
\lsptoprule
& & \textbf{Standard Japanese} & \textbf{Kagoshima Traditional} & \textbf{Kagoshima Innovative}\\
\textit{momiji} & \textbf{surface tone} & \textit{mo'miji} HLL & \textit{momiji} LLH & \textit{momi'ji} LHL\\
'autumn leaves' & \textbf{accentuation} & accented & unaccented & accented\\
\textit{momiji} & \textbf{surface tone} & \textit{kaede} LHH & \textit{kae'de} LHL & \textit{kaede} LLH\\
'maple' & \textbf{accentuation} & unaccented & accented & unaccented\\
\lspbottomrule
\end{tabular}
}
\caption{Surface tone and accentuation in SJ, Traditional KJ and Innovative KJ}
\label{tab:2}
\end{table}

In this report, following Kubozono’s reasoning, we will attempt to present additional statistical results which can specify relevant factors to this change, and consider their implications for this tonal change in terms of variation theory.

\section{Research design}

\subsection{Data collection}

[Speakers and social variables] The speech data was collected in 2011 and 2012 from 20 college students (10 males and 10 females), who were raised in the ‘Satsuma area’ of Kagoshima Prefecture. Their parents were also native KJ speakers, except for one female speaker’s parents. So it is presumable that the speakers were raised in linguistic conditions where they could pick up the traditional prosodic features. The information about speakers’ variables such as gender, hometown, and social network were obtained by employing a structured questionnaire.\footnote{ This questionnaire was made up by adopting the basic structure of the questionnaire created by the Glasgow Media Project team led by Jane Stuart-Smith. We would like to express our deepest gratitude for their cooperation.}

[Speech data] The results presented here are parts of a research project investigating prosodic innovations called Prosodic Subordination (PS) of multiple accentual phrases (MAP) (cf. \citealt{androutsopoulos_media_2014}). Since Kubozono’s results are based on the analysis of mono-stylistic productions (only in word list style), we attempted to collect more data in two other stylistic contexts. One is the task of reading sentences (RS), and the other is playing roles in scripted conversations (SC). Only eight target words with higher sonority were selected to obtain clear pictures of pitch movement. The target words consist of four place names and four general nouns as shown in \tabref{tab:3}. Each group has two 4-syllable words and two 3-syllable words. Both tasks were recorded in both SJ and KJ. In this paper, we will discuss the results of the latter variety only.

\begin{table} 
\resizebox{\textwidth}{!}{
\begin{tabular}{lllp{0.3\textwidth}p{0.3\textwidth}}
\lsptoprule
\multicolumn{5}{c}{1st noun of MAP}\\
\midrule
\textbf{syllables} & \textbf{word} & \textbf{gloss} & \textbf{tone and accentuation of traditional KJ} & \textbf{tone and accentuation of SJ}\\
\multirow{2}{.05\textwidth}{3} & \textit{Nagano} & \multirow{4}{.2\textwidth}{[place name]} & LLH unaccented & HLL accented\\
& \textit{Ueno} & & LHL accented & LHH unaccented\\
\multirow{2}{.05\textwidth}{4} & \textit{Aomori} & & LLLH unaccented & LHLL accented\\
& \textit{Miyajima} & & LLHL accented & LHHH unaccented\\
\end{tabular}
}
\vskip11pt
\resizebox{\textwidth}{!}{
\begin{tabular}{lp{0.3\textwidth}p{0.3\textwidth}p{0.3\textwidth}}
\multicolumn{4}{c}{2nd noun of MAP}\\
\midrule
\textbf{word} & \textbf{gloss} & \textbf{tone and accentuation of traditional KJ} & \textbf{tone and accentuation of SJ}\\
\textit{nomiya} & `bar' & LLH unaccented & HLL accented\\
\textit{nimono} & `stewed food' & LHL accented & LHH unaccented\\
\textit{omiyage} & `souvenir' & LLLH unaccented & LHHH unaccented\\
\textit{nizakana} & `fish cooked in broth' & LLHL accented & LHLL accented\\
\lspbottomrule
\end{tabular}
}
\caption{Target words in two tasks (RS and SC) and their tones in SJ and KJ.}
\label{tab:3}
\end{table}

\ea
\label{ex:1}
%\langinfo{lg}{fam}{src}\\
\gll Miyajima-no/de     omiyage-o    takusan   moratta\\
{Miyajima(place)\textsc{{}-gen/loc}}  {souvenirs-\textsc{obj}}   {a lot}  {receive-\textsc{past}}\\ 
\glt `I received a lot of souvenirs from/in Miyajima.'
\z

The target phrases in both tasks were formed with the combination of two phonological words. There are three components for each phrase: the first word + the case marking particle (genitive \textit{–no} or locative \textit{–de}) + the second word as shown in (1). In RS, we obtained 16 MAP phrases in KJ. For the 3-syllable version, there are eight phrases (two for the first word by two for the particle by two for the second word), and the 4-syllable version also has eight phrases. On the other hand, in SC, only 8 phrases of the genitive version are collected. In total, 48 tokens for each speaker were obtained, since each MAP phrase comprises two target words.

\subsection{Date coding}
[Dependent variables] According to \citet{kubozono_tonal_2007}, two tonal changes show an asymmetry; the change from tone B to A, from the unaccented to the accented, goes farther than the other. This is also attested in our overall results. The number of tokens for the change from Tone B to A is 322 out of 480 (67.0\%), whereas the the number in the other direction is 244 out of 480 tokens (50.8\%), including the words \textit{omiyage} and \textit{nizakana}, whose traditional accentuations are identical to that of TJ. This fact led us to speculate that each change has its own orientation. Thus, we set up two different dependent variables in terms of accentuation: One is the change from the unaccented tone (Tone B) to the accented one (Tone A); the other is the opposite, i.e., from the accented one (Tone A) to the unaccented one (Tone B). We call the former ‘accentual correspondence to the accented words of TJ (in short, ‘correspondence to accented’)’ and the latter ‘accentual correspondence to the unaccented words of TJ (in short, ‘correspondence to unaccented’)’. For the multivariate analysis (Logistic Regression), dependent variables were coded with one of the binary values, 1 or 0; when the target variant occurs, the token was coded as ‘1’, otherwise ‘0’ was given.

[Independent variables] Categorical variables include the number of syllables of target words (3 or 4), style (RS or SC), gender (male or female), and hometown (city or rural). Since they are all binary variables, the first category in the parentheses was set as a reference category in regression analyses. The case marking particles, \textit{{}-no} and \textit{–de}, mentioned in (1), were not included in multivariate analyses, because they had no effect on tonal changes in the preliminary analyses.

Quantitative variables are classified into three categories. The first indicates the possibility of taking in variation from other varieties, the other two are ‘density of current social network’ and ‘dialect contact within current social network’. Speakers were asked to list five people who they usually hang out with. The density of social network was measured by checking the degree of mutual acquaintance of the five people: A score from 1 ‘don’t know each other’ to 3 ‘know very well’ was given to each relationship. There were 10 combinations for each network, so the range of the score is from 10 to 30. Dialect contact was measured by examining the regional varieties which the five people usually used. If they use a variety other than KJ of the Satsuma area, one point was added. The total score for the five people ranged from 0 to 5.

The next category, the most significant in this report, is media content viewing habits. This is for investigating whether the broadcast media has any effect on language variation and change, which has been a long-standing controversy in sociolinguistics. This category is divided into two parts. One is watching habit of anime in the past. Twenty anime programs for children were listed and speakers were asked to give scores to each program on a five-point scale, 1 (never watched) to 5 (watched very often). Since anime is a single category of broadcasting content, we set the total score of all programs as the value of this variable. The other category is current TV programs viewing habits. Since dramas and anime for teenagers and young adults are generally broadcast only three months, it is difficult to define a specific program like \textit{East Enders} for checking speakers’ viewing habits. So we listed ten categories of TV programs, such as news, variety shows, anime, etc., and asked the speakers to score them on a five-point scale, 1 (never watch) to 5 (watch vary often). At first, we attempted to set all the categories as an individual variable, but there were correlations between them. Then, we created three new variables by Principal Component Analysis (the contribution ratio is 56.23\%): ‘watching information programs’ (news, information shows, etc.), ‘watching entertainment programs’ (variety shows, local area shows, etc.), and ‘watching pop culture programs’ (anime, music shows, etc.).  

The last variable is the one indicating speakers’ competence of SJ. As \citet[324]{kubozono_tonal_2007} refers to the bilingualism of native speakers of KJ as a bilingual listener, it would be presumable that the ongoing tonal change is deeply related to speakers’ bilingual competence of using both KJ and SJ. Thus, we set up a variable, ‘standard style score’, which consists of the total number of successful tokens in the tasks of RS and SC in SJ. The range of the value is from 19 to 48 (its maximum).

In fact, there were other social variables included in the questionnaire, such as dialect contact beyond the Kagoshima area, and speakers’ daily social practice. However, we decided to adopt the variables mentioned above by considering their theoretical relevance. 

\section{Statistical results}
Due to the asymmetry of ongoing tonal changes, we carried out two Logistic Regressions with two dependent variables; one is for the variation ‘accentual correspondence to the accented words of TJ’, the other for ‘accentual correspondence to the unaccented words of TJ’. The results are shown in \tabref{tab:4}. The statistical values for the categorical variables are indicated only for the parenthesized category. The most important value is Exp(B) indicating the probability of the target variant. When the value is greater than one, the variable has a positive effect on the occurrence of the target variant, while a value between zero and one means that the effect is negative.

For ‘correspondence with accented’ (the left half of \tabref{tab:4}), the variables having a positive effect are only ‘4 syllables’ and ‘TV1: news and information shows’. The variables with negative significance (p {\textless} .05) are ‘female’, ‘watching anime in childhood’, and ‘TV3: pop culture’. Only ‘rural’ (positive) and ‘style’ (negative) are marginally effective, and the other variables are not statistically significant. The variables concerning dialect contact do not have a clear effect on this variation. 
 
\begin{table} 
\resizebox{\textwidth}{!}{
\begin{tabular}{p{.35\textwidth}llllllll}
%\lsptoprule
& \multicolumn{4}{p{.35\textwidth}}{Accentual correspondence to the accented words of TJ\footnote{N = 480, df=1, Nagelkerke R Square: .240.}} & \multicolumn{4}{p{.35\textwidth}}{Accentual correspondence to the unaccented words of TJ\footnote{N = 480, df=1, Nagelkerke R Square: .283.}}\\
\midrule
Factors & B & Wald & Sig. & Exp(B) & B & Wald & Sig. & Exp(B)\\
\midrule
gender (female) & -1.38 & 26.84 & 0.000 & 0.252 & -0.16 & 0.37 & 0.541 & 0.856\\
hometown (rural) & 0.52 & 3.03 & 0.082 & 1.680 & -1.19 & 18.43 & 0.000 & 0.305\\
syllable (4) & 1.19 & 29.18 & 0.000 & 3.292 & 1.52 & 50.77 & 0.000 & 4.557\\
style (SC) & -0.38 & 2.89 & 0.089 & 0.682 & 0.48 & 4.74 & 0.029 & 1.618\\
density of current SN & -0.02 & 0.37 & 0.545 & 0.985 & 0.09 & 10.99 & 0.001 & 1.088\\
density of contact within current SN & -0.04 & 0.10 & 0.754 & 0.960 & 0.09 & 0.50 & 0.478 & 1.089\\
watching anime during childhood & -0.03 & 5.31 & 0.021 & 0.971 & 0.04 & 10.49 & 0.001 & 1.038\\
TV1: information programs & 0.39 & 9.93 & 0.002 & 1.472 & -0.14 & 1.49 & 0.222 & 0.869\\
TV2: entertainment programs & 0.18 & 1.77 & 0.183 & 1.193 & -0.18 & 2.16 & 0.142 & 0.837\\
TV3: pop culture programs & -0.30 & 7.53 & 0.006 & 0.739 & 0.27 & 6.22 & 0.013 & 1.310\\
standard style score & -0.01 & 0.52 & 0.469 & 0.989 & 0.07 & 17.12 & 0.000 & 1.069\\
Constant & 3.70 & 11.83 & 0.001 & 40.329 & -7.38 & 47.41 & 0.000 & 0.001\\
%\lspbottomrule
\end{tabular} 
}
\caption{Logistic Regression for accentual correspondence with TJ words.}
\label{tab:4}
\end{table} 

The results of  ‘correspondence with unaccented’ (the right half of \tabref{tab:4}) obtained more positive variables. The variables with a positive effect are ‘4 syllables’, ‘scripted conversation style’, ‘density of social network’, ‘watching anime in childhood’, ‘TV3: pop culture programs’, and ‘standard style score’. The only negative effect variable is ‘rural’. There are no marginal ones.

\section{Discussion}
A more elaborate theoretical discussion is needed, both in phonology and variation theory, to present a more conclusive and persuasive analysis. We will only suggest here a possible reasoning about the asymmetry of tonal change by considering the social meaning of this tonal variation (cf. \citealt{eckert_variation_2008}). 

Why is there a difference in the progress of change between 'correspondence with accented' and 'correspondence with unaccented'? It may be ascribed to the difference of social meaning attached to each tone. It is possible that the accented tone sounds more like SJ, because the variable 'information programs' associated with the standard variety has a positive effect on the accented tone. Thus, we can assume that this tone indexes 'normative SJ'. Its normativeness would be also supported by the negative correlation with the variable 'pop culture programs' containing novelty or deviation from the standard social norm. On the other hand, considering the positive effect of the variable 'pop culture programs', the unaccented tone seems to be associated with 'youth culture' with a flavor of reduced normativeness. Another piece of supportive (but only speculative) evidence is the spread of the flat pitch (or unaccented) tone pattern for some traditionally accented words, such as \textit{kareshi }(LHH) 'boyfriend' and \textit{baiku} (LHH) 'motorbike', since the 1990s. This innovative tone pattern seems to have spread nationwide largely through mass media broadcasting, implying a certain indexical meaning associated with the youth culture of the Tokyo metropolitan area. Considering these facts, it is assumable that these two ongoing tonal changes in KJ have different orientations; the orientation towards the accented tone is de-dialectization, whereas the other towards the unaccented tone is de-standardization. The other factors with a positive/negative effect could be incorporated into this line of reasoning, although we do not discuss their theoretical significance here.

Finally, we make a brief comment on the variables of media content viewing habits. These variables should not be regarded as a direct stimulus which can cause a speaker’s language to change. Rather, they seem to work not only as a linguistic model for exhibiting socially dominant or counter norms, but also as a circumstantial or cultural factor to provide language resources for, e.g., a speaker’s styling as well as the input for children’s language acquisition.
     
\printbibliography[heading=subbibliography,notkeyword=this]
\end{document}