\documentclass[output=paper]{LSP/langsci}   
\abstract{Given that Catalan micro-syntactic studies are still at a preliminary stage, this contribution aims to present an already-tested methodological roadmap for the study and the analysis of a particular dialect syntax variable.  Specifically, we present the different data compilation methods used to obtain fine-grained data to characterize morpho-sintactically the expression of negation in Pallarese Catalan, expressed by the post-verbal negative minimizer \textit{cap} (‘head’). In order to obtain robust evidence, the appearance of each linguistic structure is matched to one of the following data compilation strategies: exhaustive revision of Catalan dialect literature; recorded speech conversation and participant observation in speakers’ daily routines; scan reading of literature written by ‘dialect aware’ authors; grammatical judgments and ‘meta-corpus’. Data obtained is interesting both from an intra-linguistic and a cross-linguistic point of view, regarding: the relation of the marker \textit{cap} with the sentential marker \textit{no} and other negative elements (n-words, NPIs, etc.); and regarding its position and mobility in the sentence and in verbal complexes. To sum up, with this brief case study, we exemplify the idea that the survey of micro-syntactic phenomena in Catalan is a promising challenge. In effect: linguistic research may play an important role in the social acknowledgement and valuation of dialectal phenomena that are neither present in the normative language, nor in the standard variety of the media and schools.
}

\title{The future of Catalan dialects’ syntax. A case study for a methodological contribution}
\author{Ares Llop Naya\affiliation{Universitat Autònoma de Barcelona}}
% 
\maketitle 
\begin{document}

% The future of Catalan dialects’ syntax. A case study for a 
% methodological contribution\\
% \\
% Ares Llop Naya\\
 

\section{Introduction and background}

\subsection{ General remarks}
Even if Catalan has a long tradition of dialectal studies on phonology, morphology and lexicology, micro-syntactic studies still are at a preliminary stage. General properties of the language have been investigated; descriptive generalizations have been stated and some syntactic variants have been noted, but dialectal syntax phenomena are still rather unexplored –data is sparse and unsystematic and there is no dialectal syntax-oriented annotated corpus. In this sense, this contribution aims to present the methodological roadmap and strategies used for the study and the analysis of a particular dialect syntax variable.

Such kind of research looks forward to the promotion of Catalan in the well-stablished net of European dialect syntax studies and projects. Furthermore, developing an exhaustive survey on how to approach Catalan syntax variation can lead to the acknowledgement, use and attachment to dialect structures used by speakers –but reflected neither in the normative language, nor in the standard variety of the media and schools. As stated by \citet[30]{trudgill_sociolinguistic_2002}: “If we wish to maintain linguistic diversity and oppose linguistic homogenisation, we have to consider speakers’ attitudes to their own dialects. […] There is often a direct relationship between the degree of hostility to dialects, the amount of denigration of vernacular varieties, and the rate at which they disappear”. Therefore, the fruitful period of standardisation of the Catalan language developed after years of prohibition has to be followed by an attempt to preserve the richness and status of the dialects, especially peripheral and minority ones, such as Pallarese Catalan, the one we study here.

\subsection{Aim}
The aim of our research was to design, apply and test data compilation methods to obtain fine-grained data to characterize morphosintactically negative emphatic constructions in Catalan and other Pyrenean IberoRomance dialects (cf. \citealt{berns_present-day_2002} for a detailed description of the interests of studying dialectal variation regarding negation).  Specifically, we studied the expression of negation in Pallarese Catalan (the North-western Pyrenean Catalan dialect: one of the most conservative Catalan dialects, in contact with Aranese Gascon, Aragonese and French), expressed by the negative marker element \textit{cap} (‘head’) (cf. section 1.4. for further information about the element). 

\subsection{Methodological framework}
Micro-comparative syntax projects have designed a ‘layered methodology’ \citep{cornips_field_2007} to approach dialect syntax and data collection progressively and rigorously; to clearly understand the general properties of the area investigated by carrying out detailed analysis of single phenomena. This method has been chosen wisely as: on the one hand it is used in well-stablished dialect syntax projects in Europe, and on the other hand it updates and adjusts traditional data compilation methods to syntax-oriented investigations; it starts with a broad survey, and progressively narrows the target to find something interesting for micro-comparative linguistic research”.

The strong point of our proposal, in accordance with this framework, has been to develop a rigorous morphosyntactic analysis by matching each of the strategies used to compile data related to a specific variable or context. Fine-grained data can also be determinant to stablish not only cross-linguistic differences but also intra-linguistic variation. 

\subsection{Case study}
As in other Romance languages, in the Pyrenean dialects of Catalan (mainly Pallarese), emphatic negation is constructed by adding a post-verbal particle originating from a minimizer to a sentence containing the sentential negative marker equivalent to ‘no’. The singularity of this kind of reinforcement is that, originally, it was a minimizer (noun denoting a minimal amount of something). Cross-linguistically, these elements were reanalysed in negative contexts and lost their nominal value. After different stages of reanalysis they became negative polarity items, emphatic polarity particles or even markers of sentential negation, e.g. \textit{pas} ‘step’,\textit{ goutte} ‘drop’, \textit{point} ‘point’ (French), \textit{mica} ‘crumb’ (Italian), etc. (see \citealt{schwegler_word-order_1988,detges_grammaticalization_2002}). 

According to \citet{rigau_mirative_2012}, it is worth studying these kinds of particles not only as an instantiation of lexical variation, but as micro-syntactic phenomena. In this case, after a preliminary survey –and following \citet{cornips_field_2007} studies on micro-syntax and negation– we detected that in the north-western Pyrenean dialects of Catalan the behaviour of \textit{cap} was interesting cross-linguistically (cf. French and Occitan examples) and intra-linguistically (cf. standard Catalan, Central Catalan, Roussillonais Catalan examples) regarding: (1) its relation with the sentential marker \textit{no} and other negative markers; and (2) its position and mobility in the sentence and in verbal complexes. 

\begin{enumerate}
\item From a semantico-syntactic point of view, \textit{cap} can be legitimated by the sentential negative marker \textit{no }(like in standard Catalan). It can also appear without \textit{no }(like the marker \textit{pas }in colloquial negative sentences in French, and the sentential negative marker \textit{pas }in Occitan and Roussillonais Catalan)\textit{. }This might be an indication of an ongoing process of change from \textit{cap }carrying an emphatic value to expressing sentential negation by itself (cf. \textit{Jespersen’s Cycle}, \citealt{jespersen_negation_1917}; revisited in theoretical terms by \citealt{roberts_diachronic_2007,schwenter_fine_2006}). In this sense it is worth studying the relationship of negative concord with different negative quantifiers so as to determine its formal features and conditions of legitimization.

%%%decided to redo these as (hidden) tables, because I couldn't get the alignment to work otherwise
\begin{tabular}{lllllr}
(1) 
& {(Jo)} & (no) & vindré & \textbf{cap}. & {(Pall. Cat.)}\\
& {(Jo)} & no   & vindré & pas. & {(Stand. Cat.)}\\
& {(Jo)} &      & vindré & pas. & {(Rouss. Cat.)}\\
& & & & & \\
& {(Ieu)}&      & vendrai & pas. & {(Occ. Leng.)}\\
& {Je}   &{(ne)}& viendrai & pas. & {(French)}\\
& & & & & \\
&{1.sg.} & \textsc{neg} & 1.come.\textsc{fut}. & {\textsc{emph}/\textsc{neg}}. & \\
& \multicolumn{5}{l}{‘I will not come (at all).’}\\
\end{tabular}

\begin{tabular}{llllllr}
(2) 
& {(Jo)} & (no) & tornaré & \textbf{cap} & mai. & {(Pall. Cat.)}\\
& {(Jo)} & no   & tornaré & pas & mai. & {(Stand. Cat.)}\\
& {(Jo)} &      & tornaré & pas & mai. & {(Rouss. Cat.)}\\
& & & & & & \\
& {(Ieu)}&      & tornarai & pas & jamai. & {(Occ. Leng.)}\\
& {Je}   &{(ne)}& viendrai & (*pas) & jamais. & {(French)}\\
& & & & & & \\
&{1.sg.} & \textsc{neg} & 1.come.\textsc{fut}. & {\textsc{emph}/\textsc{neg}}. & never & \\
& \multicolumn{6}{l}{‘I will never come back.’}\\
\end{tabular}

\item We also decided to investigate the position of the negative marker in relation to the verb and other elements in the same syntactic position (i.e. the low IP focus position, à la \citealt{belletti_aspects_2004}). Keeping in mind that the equivalent particle in standard Catalan \textit{pas} allows mobility in certain contexts, we also analyzed the position of the particle in verbal complexes, embedded clauses, raising clauses and restructuring predicates so as to determine the degree of mobility or rigidity of the particle. 
\end{enumerate}

\subsection{Strategies used}
The five strategies used to compile data were conceived to match with a specific variable or context of interest to our research. The methods used were: 

\begin{itemize}
\item \textbf{Exhaustive revision of Catalan dialect literature}; i.e. dialect monographs, dictionaries, articles and books where the topic is touched on briefly, to compile preliminary data and prescriptive rules about the use of the element.
\item \textbf{Recorded speech conversation and participant observation in speakers’ daily routine: }semi-structured interviews\textbf{ }(à la Lebo, about anthropological topics) to find more complex examples and to explore variation depending on the register. We also recorded ‘casual speech’ contexts (outside the interview format) and we obtained data from ‘participant observation’ in daily-routine contexts, as well as from ‘out of the blue’ examples –collected unexpectedly in colloquial contexts. The main aim of this strategy was to instantiate how language change takes place earlier in more colloquial registers and in specific grammatical contexts and later spreads to other more complex constructions, cf. \citet[§3.4]{roberts_diachronic_2007}.
\item \textbf{Scan reading of literature written by “dialect aware” authors: }to find natural and real emphatic constructions in literary works written by native Catalan Pallarese speakers, who compiled oral stories and transcribed recorded conversations. We expected to attest find more complex and less common structures than the ones obtained from the previous methods in oral speech (such as periphrasis, raising, passive and factitive constructions). 
\item \textbf{Grammatical judgments:} to elicit constructions that hardly ever occur in informal speech for a fine-grained characterization of all possible contexts and interactions of \textit{cap} (periphrasis, subordinate clauses, raising constructions and clitic climbing phenomena, answers to a \textit{yes-no} question, imperative and interrogative sentences, biased questions, as a marker of constituent negation and as an expletive); and to examine negative data. The questionnaires of grammatical judgements (designed following Espinal’s works and with specially designed statements, cf. \citealt{espinal_two_1993,espinal_negacio_2002}), were written in Pallarese Catalan and presented to native speakers with a high degree of dialect awareness. The participants were advised that this was a study of dialectal forms and not standard forms of Catalan. 
\item \textbf{`Metacorpus': }following Silverstein’s works, we compiled a corpus of sociolinguistic and metalinguistic comments stated by native informants during the process of data collection. This kind of data constitutes additional information about what \citet{niedzielski_folk_1999} call \textit{folk-linguistic facts} (linguistic objects as viewed by nonlinguists).
\end{itemize}

\section{Results and discussion}
Our study explored the strong points of five different strategies to obtain the most fine-grained qualitative data to characterize morphosintactically the unexplored negative marker \textit{cap, }and to determine the range of variation and the variables of changes. We came across more than 260 occurrences from almost 30 different contexts and types of sentences (cf. \citealt{llop_negacio_2013}). The main results obtained per strategy were the following:

\begin{itemize}
\item \textbf{Exhaustive revision of Catalan dialect literature:} the results were restrictive, and examples were insufficient for a rigorous characterization of the whole picture. The variable studied was mainly considered from a lexical point of view and the most recurrent structure was the one with the preverbal marker \textit{no, }simple tense and postverbal negative marker \textit{cap}. Prescriptive grammars didn’t mention the marker studied at all and all references to emphatic negation were to comment on the standard variant \textit{pas}.
\item \textbf{Recorded speech conversation and participant observation in speakers’ daily routine: }the number of examples stated was not really high (17\%), but it was interesting that in more colloquial contexts and with simple tenses, the presence of the variable without the preverbal \textit{no} licensing the emphatic marker \textit{cap} increased significantly; it was systematic with certain verbs such as \textit{saber }‘know’. We could compile evidence in favour of an ongoing (and uncertain) process of change towards a further stage of grammaticalisation of \textit{cap,} already given for \textit{pas }in French and Occitan, languages in contact with this northern Catalan dialect
\item \textbf{Scan reading of literature written by “dialect aware” authors: }this strategy was the most productive in terms of quantity –the examples found represent the 72\% of the corpus–, and in terms of richness of constructions (periphrasis, embedded clauses, raising constructions, idioms, factitive construction, etc.). In regards to speakers, the use of complex syntactic structures with \textit{cap }can be an interesting way to emphasize its productivity and to show up and diffuse the constraints of use. We discovered, for example, that Pallarese \textit{cap }is sensitive to the “embedded negation constraint” stated for French by \citet[193]{horn_remarks_1978}, i.e.: differently to \textit{pas }in Central Catalan, Pallarese doesn’t allow the movement of the negative marker from the main clause to the embedded clause –even if blocking effects of the embedded clause boundary are neutralized by using modal constituents (cf. \citealt{espinal_two_1993,llop_negacio_2013}).
\item \textbf{Grammatical judgments:}\textbf{\textmd{the}} answers to the grammatical judgements and the degree of acceptation or rejection were very regular and uniform in all informants. Initial written questionnaires for grammatical judgements, were changed into oral ones to avoid informants being disturbed by unfamiliar written forms of dialectal variants. The most interesting discovery was the fixed position of \textit{cap }in restructuring predicates (in contrast to \textit{pas }in Central and North-western Catalan, where it is much more mobile).

%%%
%%% the example numbering is off here, because I cheated with tables earlier
%%%
\ea
\label{ex:}
%\langinfo{}{}
\gll {(Jo)} {No} \textbf{ho} {puc}   {(\texttt{\textsuperscript{\checkmark}}\textbf{cap / pas})}   {fer} {(*\textbf{cap} / \texttt{\textsuperscript{\checkmark}}\textbf{pas}).}\\
{1.sg.}  {\textsc{neg}}   {\textsc{cl}.it}  {1.\textsc{pres.mod}.can}  {\textsc{emph}/\textsc{neg}.}  {\textsc{inf}.do} {\textsc{emph}/\textsc{neg}.}\\
\glt `I can’t do it.'\\
\z

\item \textbf{Metacorpus}: we collected general comments about the structure studied, about: attitudes and sociolinguistic variables (generation divergences, comments related to the pressure of the standard language, etc.); linguistic variables (presence and absence of \textit{no}, with NPI and N-words)\textit{; }and further examples. 
\end{itemize}

\section{Conclusions}
Our contribution is an already tested methodological backbone for innovation and optimization in Catalan dialect syntax variation research. Catalan syntax variation is almost an unexplored territory which benefits from every humble initiative that promotes  its launch into the European net of syntactic studies. With the research we have briefly summarised here, we have worked on a very specific phenomenon. We have obtained evidence about: the morphosyntactic value of the negative marker \textit{cap}, its structural position from a theoretical point of view (cf. \citealt{llop_negacio_2013,llop_cap_2014}), its features and the necessity of legitimation by a negative element, etc. With this data we have looked very briefly –due to space constraints– at descriptive, historic, theoretical and micro-comparative insights. 

Developing further approaches as the one presented here we can revitalise and make visible an important amount of focalisation or emphatic dialectal particles and its semantic and pragmatic properties within the framework of the expression of negation in natural languages.

Concerning the future of our dialects, the presence of dialectal structures in standard language taught at school and in the media and creative literature can be, by far, the most influential agents for the reinforcement of dialect speakers’ awareness and self-confidence. Nevertheless, linguistic research may also play an important role in the acknowledgement and valuation of the richness of dialectal phenomena and unstudied structures such as the one presented here. In that sense, \citet[80]{rigau_variacio_1998} postulated that “the clearer we know the terms and limits of variation, the better the knowledge and use of our language will be.”  Following up, \citet[31]{trudgill_sociolinguistic_2002} stated that “linguists are in a particularly strong position to oppose this discrimination and consequent homogenization because they, as experts on language, have the knowledge and ability to engender positive attitudes and to counter the denigration”. The survey of a wider range of micro-syntactic phenomena in Catalan is a promising challenge to be followed up.

\printbibliography[heading=subbibliography,notkeyword=this]
\end{document}
